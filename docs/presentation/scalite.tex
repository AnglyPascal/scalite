\documentclass[compress, aspectratio=169]{beamer}
\usepackage[noindex, ]{ahsan_new}
\usepackage{pdfpages}

\title{Scalite}
\subtitle{A blog-aware static site generator}

\renewcommand{\codefontsize}{\color{BestBlack}\footnotesize}
\renewcommand{\commentstyle}{\color{DimGray}}

\usepackage[none]{hyphenat}

\setbeamersize{description width=.1em}
\renewcommand{\imp}[1]{\texttt{\textbf{#1}}}
\renewcommand{\note}[1]{\textcolor{NordGreen}{#1}}


\AtBeginSection[]{
  \begin{frame}[c,noframenumbering,plain]
    \tableofcontents[sectionstyle=show/hide,subsectionstyle=show/show/hide]
  \end{frame}
}

\AtBeginSubsection[]{
  \begin{frame}[c,noframenumbering,plain]
    \tableofcontents[sectionstyle=show/hide,subsectionstyle=show/shaded/hide]
  \end{frame}
}

\begin{document}

\begin{frame}[plain,noframenumbering]
  \maketitle
\end{frame}

\section{What is it?}

\begin{frame}
    Scalite is a general purpose static site generator. 

    Inputs are:
    \begin{itemize}
        \item Contents of the website
            \begin{itemize}
                \item in the form of formatted text files such as Markdown,
                    reStructuredText, Textile etc.
            \end{itemize}
            \pause
        \item Templates specifying the layout of the webpages
            \begin{itemize}
                \item rendered with the contents of text files and other parameters
            \end{itemize}
            \pause
        \item Configuration files
            \begin{itemize}
                \item YAML, JSON files specifying configurations, variables etc.
            \end{itemize}
            \pause
        \item Stylesheets: SASS, CSS etc.
    \end{itemize}
    \pause

    The output is a static website with the HTML, CSS and JS files in proper directory
    structure.
\end{frame}

\begin{frame}[fragile]{How it works}
    The user sets up a folder with the following folder structure:
\begin{lstlisting}[style=fonts]
.
+-- _assets
|   +-- website assets go here
+-- _layouts
|   +-- template files go here
+-- _plugins
|   +-- plugins go here
+-- _posts
|   +-- posts (in a blog) or other page contents go here
+-- _sass
|   +-- stylesheets go here
+-- _config.yml
\end{lstlisting}
\end{frame}

\begin{frame}[fragile]{How it works}
    Scalite then finds all the files, loads them in the runtime. It then
    \begin{itemize}
        \item Reads the configurations
            \begin{itemize}
                \item Loads the specified plugins
            \end{itemize}
            \pause
        \item Converts all formatted text files into HTML files
            \begin{itemize}
                \item Users can provide converter plugins to support new formats 
            \end{itemize}
            \pause
        \item Create \imp{Page} objects, and 
            \begin{itemize}
                \item handles cross referencing
                \item creates categories and tags
                \item and other features the user specifies
            \end{itemize}
            \pause
        \item For each specified page, loads its template, converted contents and
            compiles the files into final HTML files
            \pause
        \item Creates the destination folder, copies the pages and assets to the
            destination folder
    \end{itemize}
\end{frame}

\begin{frame}[fragile]
    \vspace{-1.4em}
    \begin{minipage}[t]{.47\linewidth}
    \begin{lstlisting}[style=mymd, caption={hello_world.md}]
---
# YAML header
# local variables, configs
title: Front Page
tag: tag1, tag2
---

Hello, **World**!
    \end{lstlisting}
    \end{minipage}\hfill%
    \begin{minipage}[t]{.47\linewidth}
    \begin{lstlisting}[style=myhtml, caption={page.mustache}]
<html>
  <body>
    <header>
      {{ title }}
    </header>
    {{> content}}
  </body>
</html>
    \end{lstlisting}
    \end{minipage}

    \begin{lstlisting}[style=myhtml, caption={hello_world.html}]
<html>
  <body>
    <header>
      Front Page
    </header>
    Hello, <b>World</b>!
  </body>
</html>
    \end{lstlisting}
\end{frame}

\begin{frame}{Motivation}
    Scalite is inspired from \textbf{Jekyll}. It's is an attempt to recreate Jekyll
    while generalizing many of its features. 

    Scalite attempts to identify the core components, and make them as loosely
    coupled as possible. 
\end{frame}

\begin{frame}{Motivation}
    Scalite's core goals are:
    \begin{itemize}[<+->]
        \item To be language agnostic:
            \begin{itemize}
                \item it should support contents written in any markup language
                \item it should support any templating language
            \end{itemize}
        \item To be easy to use, configurable, and flexible
            \begin{itemize}
                \item all the features should be customizable through config files
            \end{itemize}
        \item To be extensible and fully customizable
            \begin{itemize}
                \item plugins should have a powerful API
            \end{itemize}
        \item It should never assume any particular website structure
            \begin{itemize}
                \item the website structure should be easily modifyable
            \end{itemize}
    \end{itemize}
\end{frame}

\begin{frame}{Sacrifices}
    However, flexibility comes at a cost,
    \begin{itemize}[<+->]
        \item Implementation becomes exponentially complex
            \begin{itemize}
                \item strong type system of Scala makes it more complicated
            \end{itemize}
        \item Class definitions are generalized to be loaded at runtime to simplify
            plugin implementation
            \begin{itemize}
                \item But this incurs heavy runtime overhead
            \end{itemize}
        \item Great deal of consideration is needed to define what plugins can and
            can't control
            \begin{itemize}
                \item There might be data leakage bugs unless immutable data structures
                    are used throughout
            \end{itemize}
    \end{itemize}
\end{frame}

\section{Project Structure}

\begin{frame}[fragile]{Project structure: Basis}
    Basis modules that are more or less used by all other modules:
    \begin{itemize}
        \item \imp{documents}: module defining interfaces for common features 
            \begin{itemize}
                \item \imp{Assets}: Class to handle asset files
                \item \imp{Convertible}: files that need to be converted to some
                    other format
                \item \imp{Page, Renderable, SourceFile}: Mixins defining properties
            \end{itemize}
            \pause
        \item \imp{data}: module defining JSON-like data structures
            \begin{itemize}
                \item \imp{immutable}: read-only JSON-like structures. 
                    \begin{itemize}
                        \item communicating with external plugins 
                        \item storing global variables
                    \end{itemize}
                \item \imp{mutable}: mutable JSON-like structures
            \end{itemize}
            \pause
        \item \imp{util}: utility functions, parsers, logging mechanism etc.
    \end{itemize}
\end{frame}

\begin{frame}[fragile]{Project structure: Convertibles}
    \begin{itemize}[<+->]
        \item \imp{converters}: module defining source file -> html logic 
            \begin{itemize}
                \item \imp{Converter}: defines \imp{Converter} interface
                \item \imp{Converters}: singleton object contianing all
                    converters available at runtime
            \end{itemize}
            % \pause
        \item \imp{layouts}: module defining template files
            \begin{itemize}
                \item \imp{Layout}: Interface for a single template
                \item \imp{LayoutGroup}: Set of templates of the same language
                    \begin{itemize}
                        \item custom \imp{LayoutGroup}s for other
                            languages via plugins
                    \end{itemize}
                \item \imp{Layouts}: Singleton object with all layouts of various
                    languages available at runtime
            \end{itemize}

    \end{itemize}
\end{frame}

\begin{frame}{Project Structure: Website content}
    Contents of the website are defined as \texttt{Collection}s of \texttt{Element}s.
    \begin{itemize}[<+->]
        \item \imp{collections}: collection of contents
            \begin{itemize}
                \item \imp{Element}: interface of an element 
                    \begin{itemize}
                        \item \imp{PostLike}: blog-post like contents 
                        \item \imp{PageLike}: static page like content, i.e.
                            \imp{about, index} pages
                        \item \imp{ItemLike}: non-page elements to be inserted in
                            other elements
                        \item \imp{user_defined}: custom element styles via plugins
                    \end{itemize}
                    \pause
                \item \imp{Collection}: a single collection, typically sourced from a
                    single folder. 
                \item \imp{Collections}: Singleton object holding all available
                    collections
            \end{itemize}
    \end{itemize}
\end{frame}

\begin{frame}{Project structure: Categorizing}
    In any website, webpages are stored in the device in a tree like structure.
    Similarly, blogs typically have categories, tags etc. to structure the posts. We
    generalize this notion using \texttt{trees} module:
    \begin{itemize}[<+->]
        \item \imp{trees}: module defining structure of the pages
            \begin{itemize}
                \item \imp{Tree}: Interface defining nodes of a \texttt{Tree} containing
                    contents
                \item \imp{AnyTree}: A generic implementation of \texttt{Tree}.
                    \begin{itemize}
                        \item \imp{CategoryStyle}: defines a category style tree
                            structure
                        \item \imp{TagStyle}: defines shallow trees for Tags
                    \end{itemize}
                \item \imp{Forest}: A set of specific type of \texttt{Tree}s
                \item \imp{PostForests}: Singleton object holding all \texttt{Forest}s
            \end{itemize}
    \end{itemize}
\end{frame}

\begin{frame}[fragile]{Project structure: Categorizing}
    \begin{minipage}[t]{.48\linewidth}
\begin{lstlisting}[style=fonts]
             tags
       _______|_______
       |      |      |
      tag1  tag4   tag5
\end{lstlisting}
    \end{minipage}\hfill%
    \begin{minipage}[t]{.48\linewidth}
\begin{lstlisting}[style=fonts]
           categories
     __________|__________
     |         |         |
    cat1     cat4      cat5
  ___|___                |      
  |     |              cat6 
 cat2  cat3 
\end{lstlisting}
    \end{minipage}
    
    Here, each post, page or any other element can belong to any number of tag
    \texttt{tag}\(_{i}\) and any number of category \texttt{cat}\(_{i}\).
    \pause

    If the user wishes, these tree nodes can be rendered into seperate webpages to make
    navigation in the website easier.
\end{frame}

\begin{frame}{Project structure: Plugins and Hooks}
    \begin{itemize}
        \item \imp{Plugin}: An interface for objects that can be provided as plugins
            \begin{itemize}
                \item \imp{PluginManager}: Loads all specified plugins from
                    \texttt{/_plugins}
            \end{itemize}
            \pause
        \item \imp{hooks}: define runtime hooks for fine-grained control. 
            \begin{itemize}
                \item Each module has places where hook plugins can anchor to and
                    modify the data-flow \pause
                \item For example, 
                    \begin{itemize}
                        \item before or after reading source files,
                        \item before and after conversion, 
                        \item before or after rending templates etc.
                    \end{itemize}
                    \pause
                \item Hooks are implemented in Publisher-Subscriber pattern \pause
                \item They are added to the objects at runtime by the
                    \texttt{PluginManager}.
            \end{itemize}
    \end{itemize}
\end{frame}

\begin{frame}[fragile]{Project structure: Config file}
    The config file \texttt{/_config.yml} allows the user to set global configurations.
    It has the structure
\begin{lstlisting}[style=myyml]
#-- Global settings and variables --#
title: My site  # variable
show_excerpts: false  # setting

#-- Module settings --#
collections: #-- settings to be passed to Collections object --#
    posts: true
    articles: 
        output: true
        folder: /_articles
plugins:
    textile: #-- settings to be passed to textile plugin --#
\end{lstlisting}
\end{frame}

\begin{frame}{Project structure: Putting it together}
    \imp{Site} class at the root of the project puts all of the components together:
    \begin{itemize}
        \item Load global configurations from \texttt{/_config.yml}
        \item Send configurations to each module and load plugins
        \item \texttt{build()} command processes all the modules 
    \end{itemize}
\end{frame}

\begin{frame}{Libraries}
    \begin{itemize}
        \item \imp{weePickle}: To read configurations from YAML files and create
            internal repsentations of the data
        \item \imp{nscala-time}: To handle date-time calculation
        \item \imp{scala-parallel-collections}: For concurrency
        \item \imp{scala-uri}: To parse/simplify url/uri
        \item \imp{laika}: For the default Markdown->HTML converter
        \item \imp{scala-logging} and \imp{logback-classic}: To support logging
        \item \imp{scala-mustache}: The default compiler for mustache templates
    \end{itemize}
\end{frame}

\section{Implementation Details}

\begin{frame}{Design patterns}
    \begin{itemize}
        \item Modules are written in builder design pattern. 
            \begin{itemize}
                \item The constructors for the objects are specified at the runtime,
                    based on the plugins provided, or the configuration files
            \end{itemize}
        \pause
        \item Work is done by singleton objects, that
            \begin{itemize}
                \item receive configurations
                \item set up the constructors
                \item fetches and arranges files
                \item compilers/renders files
                \item writes them back to the disk
            \end{itemize}
    \end{itemize}
\end{frame}

\begin{frame}[fragile]{\texttt{Collections}}
    \begin{lstlisting}[style=myscala]
object Collections extends Configurable with Generator:
  /** section in the configs */
  val sectionName: String = "collections"

  /** Avaiable Element styles */
  private val styles = LinkedHashMap[String, ElemConstructor](
    "post" -> PostConstructor,
    "page" -> PageConstructor,
    "item" -> ItemConstructor
  )

  /** Plugins may add more constructors to this table */
  def addStyle(elemCons: ElemConstructor): Unit = ...
  ...
    \end{lstlisting}
\end{frame}

\begin{frame}[fragile]{\texttt{Collections}: \texttt{apply}}
    \vspace{.5em}
    \begin{lstlisting}[style=myscala]
  ...
  private val collections = ListBuffer[Collection]()
  
  /** process all collections and writes to destination */
  def process(dryRun: Boolean = false): Unit =
    for col <- collections.par do col.process(dryRun)

  /** Gets configuration set in "collections" section of
    * '_configs.yml' and creates necessary Collection objects */
  def apply(_configs: MObj, globals: IObj): Unit =
    // update method is provided by data module
    val configs = defaultConfigs update _configs 

    // get values from the configs
    val base = /** base directory **/
    val colsDir = /** relative directory where collections are */
    \end{lstlisting}
\end{frame}

\begin{frame}[fragile]{\texttt{Collections}: \texttt{apply} cont.}
    \vspace{.5em}
    \begin{lstlisting}[style=myscala]
    // create collection for each name in collecionsDir
    for (name, config) <- configs do
      config match
        case config: MObj =>
          // what kind of elements we want to make
          val style = config.extractOrElse("style")("item")
          val output = /** write the elements to destination? */
          if !output then logger.debug(s"won't output ${RED(name)}")
          else
            val dir = /** absolute directory of elements */
            val Col = 
              Collection( styles(style), // element constructor
                          name, dir, configs, globals )
            /** handle item collections so that other collections 
              * have access to the items */
            collections += Col // add to the collections map
        case _ => /** log error */
    \end{lstlisting}
\end{frame}

\begin{frame}[fragile]{\texttt{Collection}}
    \vspace{.5em}
    \begin{lstlisting}[style=myscala]
class Collection(
    private val elemCons: ElemConstructor, // element constructor
    val name: String, // name of collection
    private val directory: String, // absolute directory 
    _configs: MObj, // configs for this collection
    protected val globals: IObj // global configs and variables
) extends Renderable with Page:

  /** fetch all items of this collection */
  lazy val items: Map[String, Element] =
    lazy val constructor =  elemCons(name) // create the constructor
    val files = getListOfFilepaths(directory)

    def f(fn: String) = /** construct fn -> Element object pair */
    files.filter(Converters.hasConverter).map(f).toMap
    \end{lstlisting}
\end{frame}

\begin{frame}[fragile]{\texttt{Collection}: \texttt{process}}
    \begin{lstlisting}[style=myscala]
  protected[collections] def process(dryrun: Boolean = false) =
    for item <- items.values do
      item match
        case item: Page => item.write(dryrun) // only write Pages
        case _          => ()
    write(dryrun) // write the index page of this collection
    CollectionHooks.afterWrites(globals)(this)  
    // run after write hooks attached to the collections
    \end{lstlisting}
\end{frame}

\begin{frame}[fragile]{\texttt{Page: write}}
    \begin{lstlisting}[style=myscala]
trait Page:
  this: Renderable =>

  def write(dryRun: Boolean = false): Unit =
    if !visible then return
    val path = /** destination */

    if !dryRun then
      val up = PageHooks.beforeRenders(globals)(locals)
      val str = render(IObj(up))
      val r = PageHooks.afterRenders(globals)(locals, str)
      writeTo(path, r)
      PageHooks.afterWrites(globals)(this)
    else return
    \end{lstlisting}
\end{frame}

% \section{Libraries used}


\end{document}
